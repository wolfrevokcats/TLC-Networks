\chapter{Internet}
The Internet is a global system of interconnected computer networks that use the standard Internet
protocol suite (TCP/IP) to serve several billion users worldwide. We can refer to it as a \textit{Network of Networks}.
\section{Overview}
\begin{itemize}
\item 1961-64: Leonard Kleinrock, a brilliant MIT student, proves that packet switching is very efficient in presence of bursty traffic
\item 1967: Lawrence Roberts $\rightarrow$ Interface Message Processor (IMP): 
\begin{equation}
\begin{cases}
\text{Aim} \rightarrow \text{Interconnect \textbf{heterogeneous host computers through special nodes}}\\
\text{Fundation} \rightarrow \text{Advanced Research Projects Agency (ARPA)}
\end{cases}
\end{equation}
(ARPA) $\in$ Department Of Defense (DoD) of USA $\rightarrow$ dawn of ARPAnet
\item 1969: ARPANET is up and running
\begin{equation}
\begin{cases}
\text{4 nodes}\rightarrow \text{UCLA, UCSB, Stanford Research Inst. (SRI), Utha Univ.}\\
\text{One single protocol} \rightarrow \text{Network Control Protocol (NCP): RFC 001} 
\end{cases}
\end{equation}
The first telnet from UCLA to SRI crashes the system$\rightarrow$dawn of the demo effect
\item 1970: ALOHAnet $\rightarrow$ A wireless network to interconnect campuses of Hawaii islands
\item 1970-1980: proliferation of different (\textbf{not interconnected) heterogeneous networks}
\item 1972: Vint Cerf $\&$ Bob Kahn advance the idea of using special devices, called \textit{Gateway,} to interconnect heterogeneous networks
\begin{itemize}
\item Issues: different packet sizes, interfaces, transmit rates, reliability $\rightarrow$ mess
\item Gateways act as \textbf{universal translator} between heterogeneous networks
\end{itemize}
\item A reliable transport protocol, called \textit{Transmission Control Protocol} $\rightarrow$ 
\begin{equation}
\begin{cases}
\text{manage \textbf{e2e communications} over such a heterogeneous mix of networks}\\
\textbf{\text{TCP shifts the burden of error control and}}\\\text{\textbf{recovery from IMP to end hosts}}
\end{cases}
\end{equation}
$\Rightarrow$This is the key for the Internet scalability
\item 1977: ARPANET, ALOHANET, Packet Radio network are interconnected as ARPA Internet

$\rightarrow$ Cerf and Kahn proposed to split TCP into TCP $\&$ IP
\item 1981: UNIX includes the TCP/IP protocols
\item 1983: ARPANET (with more than 200 nodes) replaces NCP with TCP/IP 
\item National Science Foundation funded CSNET to link computer science departments
across the country and connects to ARPANET through TCP/IP
\item 1983: Birth of the \textit{Domain Name Server (DNS)} system
\item 1990: ARPANET is decommissioned
\item 1994 NFSNET is decommissioned
\textbf{\item 1995: the INTERNET becomes fully commercial}
\item Today, more than 40,000 networks are interconnected by means of TCP/IP
protocols... 40 years back by Cerf and Kahn (ah.. golden ’70!)
\end{itemize}

\section{Internet Service Provider (ISP)}
\begin{itemize}
\item Inside of a ISP
\begin{itemize}
\item Huge network (company) that collects other smaller ones
\item 2 types of Gateways: special devices to interconnect heterogeneous networks
\begin{equation}
\begin{cases}
IG \rightarrow \text{Internal Gateway}\\
EG \rightarrow \text{Exterior Gateway}
\end{cases}
\end{equation}
\end{itemize}
\item ISPs are connected in a quasi "hierarchical" manner
\begin{equation}
\begin{cases}
\text{ National ISP} \stackrel{Physical Link}{\longrightarrow} EG \in \text{$\{$Regional ISPs, Local ISPs$\}$}\\
\text{ Regional ISPs} \stackrel{Logical Link}{\longrightarrow}
\end{cases}
\end{equation}
\end{itemize}


