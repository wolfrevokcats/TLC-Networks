\chapter{Addressing Fundamentals}
An address uniquely identify the source and final destination and specify the route through the networks.
\section{Internet naming and addressing}
\subsection{Requirements}
\begin{itemize}
\item Uniqueness/unambiguity
\begin{equation}
\begin{cases}
\text{1. Every address} \stackrel{shall\, be}{\Rightarrow} \text{ associated to a } \textbf{single} \text{ entity in the network}\\
\text{2. The \textbf{same} entity} \stackrel{may\, be}{\Rightarrow} \text{ multiple addresses}
\end{cases}
\end{equation}
\item Multilevel structure
\begin{equation}
\begin{cases}
\text{1. Allow for hierarchical routing}\\
\text{2. Address shall be \textbf{unique within it scope}}
\end{cases}
\end{equation}
\item Internet $\Rightarrow$ Each address must be unique \textbf{within the scope of the layer}
\end{itemize}

\subsection{Address Level}
\begin{itemize}
\item PHY/MAC address
	\begin{enumerate}
	\item They \textbf{uniquely identify} the 				\textit{Network Interface Card (NIC)} in a local 		network.\\
	\item The strict condition is that \textbf{it must be 		unique at local level}, but in practice we can find 		it is also at global level.
    \item MAC address is \emph{Hardware} written to have a faster process
	\item \emph{Example: } Bluetooth [20-c9-d0-d5-aa-be], 		WiFi [20:c9:d0:d5:aa:bd], Ethernet [a0:4f:d4:4:91:cf]
\end{enumerate}
\item Data Link Layer address (DLL) $\Rightarrow$ Logical Link Control (LLC) address
	\begin{enumerate}
	\item They uniquely identify a logical \textit{point-to-point} connection between two directly connected hosts.
    \item \textbf{Must be unique for reach point-to-point connection}
    \item Can be repeated in different connections
    \item Usually quite short $\rightarrow$ one/two bytes
	\end{enumerate}
\item Network Layer address $\Rightarrow$IP address
	\begin{enumerate}
	\item Uniquely identify a host in the entire Internet
    \item \textbf{Must be unique at global level}
    \item \emph{Example: } only valid network address today $\rightarrow$ IP address: 192.168.13
	\end{enumerate}
\item Transport Layer Address $\Rightarrow$ Port Numbers
	\begin{enumerate}
	\item \textbf{Unique within a host for a given transport protocol}
    \item Ca be re-used in different hosts or in the sam host for different transport protocols $\Rightarrow$ TCP and UDP can re-use the same port numbers
    \item Usually \textit{Transport Layer Address = 2 bytes}
    \item \emph{Examples: } Port Number 21, 22, 23, 25, 53, 80 corresponding to \textit{File Transfer Protocol (FTP), Secure Shell (SSH), Telnet remote login service, Simple Mail Transfer Protocol (SMTP), Domain Name System (DNS) Service, Hypertext Transfer Protocol}
	\end{enumerate}
\item Application Address
	\begin{enumerate}
	\item Human readable mnemonic addresses
    \item \textbf{Unique at global level}
    \item \emph{Examples: } URL or email like www.unipd.it and zanella@dei.unipd.it
	\end{enumerate}
\end{itemize}

\subsection{Address Translation Mechanisms}
One host interface has 3 different addresses: a \emph{Host Name}, readable like "zanella.dei.unipd.it", converted to the \emph{Internet address (IP)} by the \emph{Domain Name System (DNS)} and the conversion of the \textit{IP address} into the \textit{MAC address} due to the \textit{Address Resolution Protocol} (ARP).

\subsection{Addressing in Internet}
\begin{itemize}
\item \begin{equation}{\text{ IP Address = }}
\begin{cases}
\text{ 32 bits}\Rightarrow\text{ Grouped in octects}\\
\text{ Every octects}\Rightarrow\text{ Decimal number} \in \{0,255\}\\
\text{ Every decimal number is separated by dots}
\end{cases}
\end{equation} 
\item IP address = [ IP Network Address $\mid$ IP Host Address ]

$\curarr$ \textbf{IP Network Address} \textbf{$\Rightarrow$} \textbf{Uniquely identify a \emph{Network} in the \emph{Internet}}

$\curarr$ \textbf{ IP Host Address $\Rightarrow$ Uniquely identify a \emph{Host} in a \emph{Network}}
\item \begin{definition}{\textbf{Socket}\\}
A Socket is the combination of the IP address ($\Rightarrow$ Network Layer) and the Port Number ($\Rightarrow$ Transport Layer).

$\curarr$ It represent the \textbf{endpoint of each connection in the internet}: <IP,port>

\end{definition}
\item \begin{equation}
\begin{cases}
\text{ Street} \Longleftrightarrow \text{ IP Network Address}\\
\text{ Building} \Longleftrightarrow \text{ IP Host Address}\\
\text{ Apartment} \Longleftrightarrow \text{ Port Number}\\
\text{ Person} \Longleftrightarrow \text{ One application per port number}
\end{cases}
\end{equation}
\end{itemize}


\section{Transport Service}
There are two types of \emph{Transport Service Protocols}: TCP (Transmission Control Protocol) and UDP (User Data Protocol)

\begin{itemize}
\item TCP: 
\begin{enumerate}
\item \textbf{Connection-oriented transport service} $\Rightarrow$ Host-to-Host connectivity at the Transport Layer of the Internet Mode
\item \textbf{A logical e2e connection is established between the sender and the receiver} processes and is \emph{full duplex}
\item Each endpoint is identified by \emph{<IP,port>}
\item The TCP connection is identified by the sockets of its endpoints
\item 
\begin{definition}{\textbf{TCP socket}\\}
A TCP socket is a pair of sockets as <$IP_A$,$port_A$><$IP_B$,$port_B$>
\end{definition}
\item The Transmission Control Protocol provides a communication service at an intermediate level between an application program and the Internet Protocol.
\item An application does not need to know the particular mechanisms for sending data via a link to another host, such as the required packet fragmentation on the transmission medium
\item \textbf{At the transport layer} of the protocol stack $\Rightarrow$ TCP:
\begin{enumerate}
\item Handles all handshaking + transmission details
\item Presents an abstraction of the network connection to the application
\end{enumerate}
\item \textbf{At the lower levels} of the protocol stack, $\Rightarrow$ due to \textit{network congestion, traffic load balancing, or other unpredictable network behaviour}, IP packets may be lost, duplicated, or delivered out of order $\Rightarrow$ TCP:
\begin{enumerate}
\item Detects these problems
\item Requests re-transmission of lost data
\item Rearranges out-of-order data 
\item Helps minimize Network congestion to reduce the occurrence of the other problems
\item If the data still remains undelivered the source is notified of this failure 
\item Once the TCP receiver has reassembled the sequence of octets originally transmitted $\Rightarrow$ It passes them to the receiving application
\item AT the end TCP abstracts the application's communication from the underlying networking details
\end{enumerate}
\end{enumerate}
\item UDP:
\begin{enumerate}
\item Connectionless transport service
\item There's \textbf{no logical e2e connection}
\item Sender sends the packet to a receiver and that's it
\item it suffices to specify the receiver socket $<IP,port>$
\item
\begin{definition}{\textbf{UDP Socket}\\}
A UDP Socket is just purely <$IP_B$,$port_B$>
\end{definition}
\end{enumerate}
\end{itemize}

\section{Key points}
\begin{equation}{\text{One Host}\rightarrow}
\begin{cases}
\text{ At least 1 IP address} \Rightarrow \text{\emph{A Host can have multiple IP addresses}}\\
\text{ Multiple Processes} \Rightarrow \text{Each process is identified by one port number}
\end{cases}
\end{equation}
\begin{equation}{\text{IP address}\rightarrow}
\begin{cases}
\text{\emph{IP address must be unique in the Network Domain}}\\
\curarr \text{ 		Public Internet} \rightarrow \text{Globally unique IP address}\\
\curarr \text{ 		Private Network} \rightarrow \text{Unique in the private network \textbf{only}}
\end{cases}
\end{equation}

\begin{itemize}
\item The connection is identified by one or more pairs \textit{<IP address,port number>} called \textbf{socket} which are \emph{unique in the network domain}.
\item IP addresses uniquely identify network endpoints int the Internet, except in case of anycast routing
	\begin{itemize}
	\item A public IP address can be associated to  single network endpoint except in case of anycast routing\\$\curarr$
Private IP addresses can be freely re-used in private networks, but cannot be used to route packet through the public Internet    
    \item A single network endpoint can be associated to multiple IP addresses
    \item Host connected to multiple networks \textbf{must} have multiple IP addresses
	\end{itemize}
\item Port numbers uniquely identify an application ina host
\begin{itemize}
\item Well-known port number are assigned to common applications
\item other port numbers can be used to private applications
\end{itemize}
\item A socket \textit{<IP address,Port number>} uniquely identify an endpoint of a logical connection in Internet 
\end{itemize}


